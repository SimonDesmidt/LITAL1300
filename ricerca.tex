\documentclass[12pt]{article}

\usepackage{amsfonts,amsmath,amssymb}
\DeclareMathAlphabet{\mathpzc}{OT1}{pzc}{m}{it}
\usepackage[italian]{babel}
\usepackage{caption}
\usepackage{enumitem}
\usepackage[a4paper,left=1.5cm,right=1.5cm,top=2.5cm,bottom=2.5cm]{geometry}
\usepackage{float}
\pagestyle{plain}
\usepackage{geometry}
\usepackage{graphicx}
\usepackage{hyperref}
\usepackage{setspace}
\hypersetup{
    colorlinks=true,
    citecolor=Bittersweet,
    filecolor=black,     
    linkcolor=Bittersweet,
    urlcolor= Bittersweet
}
\usepackage{mathtools}
\usepackage{nicematrix}
\usepackage{subfigure}
\usepackage{titling}
\onehalfspacing 

\bibliographystyle{alpha}

\begin{document}
%%%%%%%%%%% TITLE %%%%%%%%%%%%%%%%%
\noindent LITAL1300 - Ricerca dell'etimologia di tre parole\\[-2mm]
\rule{\linewidth}{0.5pt}

\begin{center}
    {\large LITAL1300 -- Ricerca} \\
    \bigskip 
    Simon Desmidt \bigskip \\
    \today
    \smallskip
\end{center}
\hrule
\bigskip
%%%%%%%%%%% END TITLE %%%%%%%%%%%%%

\section*{Tifosi -- per la cronica}
Oggi vorrei portarvi in un piccolo viaggio etimologico, dentro una parola che tutti gli appassionati usano, spesso senza pensarci troppo: “tifosi”. Una parola che in Italia è praticamente sinonimo di passione sportiva… ma che nasconde una storia curiosa, e un po' sorprendente.

Per capire da dove arriva, dobbiamo fare un passo indietro nel tempo. “Tifosi” deriva dal greco typhos, che significa fumo o ardore. Da questa radice nasce anche tifo, la malattia infettiva che provoca febbre altissima e uno stato di agitazione quasi delirante.

Tuttavia, proprio l'idea del delirio febbrile, di un calore che ti prende e non ti lascia, ha dato vita al significato moderno. All'inizio del ventesimo secolo, nel linguaggio sportivo italiano, si comincia a parlare di “tifo” per descrivere l'entusiasmo che montava attorno agli stadi. E così, per indicare chi partecipava con fervore quasi “febbrile”, ecco comparire la parola “tifoso”. Non come malato… ma come lui che ha la febbre della passione.

È un'immagine forte, quasi poetica: essere tifosi significa avere dentro un'energia che brucia, un'emozione che sale come la temperatura. Ed è così che la parola è entrata nel linguaggio, fino a diventare parte dell'identità culturale italiana.

Ma c'è un gruppo di tifosi che porta questo nome con un orgoglio particolare: i tifosi della Ferrari.
E qui, lo ammetto, entro in gioco anch'io\dots perché sì, sono uno di loro.

Noi ferraristi non abbiamo solo la passione: abbiamo qualcosa che somiglia davvero a una febbre. Una febbre che ti prende fin da quando senti il rombo di un motore, o vedi quel rosso inconfondibile, il rosso che è la colore del mio cuore. Essere tifosi della Ferrari significa vivere ogni gara con un'intensità che oscilla tra la gioia e la sofferenza, tra l'estasi e il tormento… un po' come un sintomo, ma che non vuoi assolutamente guarire.

E, alla fine, forse è proprio questo che rende bellissima la parola “tifosi”:
non descrive solo un pubblico, ma un sentimento.

E nel mio caso… un sentimento rosso Ferrari.
\section*{Nostalgia}
La nostalgia e un'emozione che noi Tifosi di Ferrari, gli ferraristi, consciamo bene. Sentiamo la nostalgia della vittoria. L'ultimo mondiale è del 2008, e da abbiamo avuto molto speranza, ma non abbiamo vinto mai. C'erano opportunità, certo, ma gli altri sono sempre più forti. Ecco perché oggi vorrei parlare della parola Nostalgia.\\
La parola ha un'origine interessante: viene dal greco "nóstos" (ritorno), e "álgos" (dolore). Letteralmente significa “il dolore del ritorno”, e era usata in passato per descrivere un profondo desiderio e dolore per il ritorno a luoghi, persone o tempi del passato, che non si possono più rivivere. La parola è usato durante il diciassettesimo secolo per gli medici per descrivere una patologia degli soldati svizzeri lontani da casa. Oggi non è più una malattia, ma un'emozione universale: il desiderio di ritrovare qualcosa che abbiamo amato.

Ma la nostra nostalgia non è tristezza: è carburante. È ciò che ci fa credere che prima o poi torneremo a vincere. È un legame profondo con il passato che ci spinge verso il futuro.

E forse è proprio questo il bello: se c'è nostalgia, c'è speranza e passione.
\section*{Adagio}
Ma la corsa non è la mia sola passione. Oltre ai motori, c'è un'altra cosa che accompagna la mia vita: la musica. E qui arriva un'altra parola dal fascino speciale: adagio.

In musica, adagio indica un tempo lento, disteso, quasi sospeso. È un invito a respirare, a lasciar andare la fretta. La parola viene dall'italiano “ad agio” che significa “con calma”. Non è solo una velocità: è un atteggiamento. È il modo in cui la musica ci ricorda che non tutto deve essere una corsa.

E forse, in fondo, c'è un legame con ciò che dicevamo prima. La nostalgia, il tifo, l'attesa di una vittoria che manca… tutto questo ha il ritmo di un adagio: lento, paziente, ma pieno di emozione. Perché anche quando aspettiamo, anche quando sembriamo fermi, dentro di noi la musica continua a muoversi.

E allora sì: sono tifoso Ferrari. Ma sono anche un amante dell'adagio. Due centri di interesse diversi, ma uniti dalla stessa cosa: il tempo. Il tempo dell'attesa.
\end{document}
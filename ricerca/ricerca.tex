\documentclass[12pt]{article}

\usepackage{amsfonts,amsmath,amssymb}
\DeclareMathAlphabet{\mathpzc}{OT1}{pzc}{m}{it}
\usepackage[italian]{babel}
\usepackage{caption}
\usepackage{enumitem}
\usepackage[a4paper,left=1.5cm,right=1.5cm,top=2.5cm,bottom=2.5cm]{geometry}
\usepackage{float}
\pagestyle{plain}
\usepackage{geometry}
\usepackage{graphicx}
\usepackage{hyperref}
\usepackage{setspace}
\hypersetup{
    colorlinks=true,
    citecolor=Bittersweet,
    filecolor=black,     
    linkcolor=Bittersweet,
    urlcolor= Bittersweet
}
\usepackage{mathtools}
\usepackage{nicematrix}
\usepackage{subfigure}
\usepackage{titling}
\onehalfspacing 

\bibliographystyle{alpha}

\begin{document}
%%%%%%%%%%% TITLE %%%%%%%%%%%%%%%%%
\noindent LITAL1300 - Ricerca dell'etimologia di tre parole\\[-2mm]
\rule{\linewidth}{0.5pt}

\begin{center}
    {\large LITAL1300 -- Ricerca} \\
    \bigskip 
    Simon Desmidt \bigskip \\
    \today
    \smallskip
\end{center}
\hrule
\bigskip
%%%%%%%%%%% END TITLE %%%%%%%%%%%%%

\section*{Tifosi e un po' di nostalgia -- per la cronaca}
Oggi vorrei fare un piccolo viaggio etimologico nella storia di una parola che tutti gli appassionati usano: “tifosi”, in Italia eppure nasconde una storia curiosa.

Tifoso deriva dal greco typhos, che significa fumo o calore. Da questa radice deriva anche “tifo”, la malattia che fa venire una febbre molto alta e anche un po’ di confusione.

Proprio questa idea del delirio febbrile ha dato poi il significato moderno. All’inizio del ventesimo secolo, nel linguaggio sportivo italiano, si comincia a usare “tifo” per esprimere l'entusiasmo che montava negli stadi. E così nasce la parola “tifoso”: non una persona malata, ma una persona che ha una “febbre” di passione.

È un’immagine molto forte: essere tifosi significa avere dentro un’emozione che brucia, che aumenta come la temperatura.

Un gruppo di tifosi porta questa etichetta con un orgoglio particolare: i tifosi della Ferrari.
E qui, lo ammetto, entro in gioco anch'io\dots perché sì, faccio parte di loro.

Noi ferraristi abbiamo questa passione: questa specie di febbre vera. Ci prende quando sentiamo il rumore del motore, o vediamo quel rosso unico, rosso come il colore del mio cuore. Essere tifosi della Ferrari significa vivere ogni gara con un'intensità che oscilla tra la gioia e la sofferenza, tra l'estasi e il tormento… un po' come un sintomo, ma di cui non vuoi assolutamente guarire.

Per tutto ciò, in questo fine d'anno, auguro a tutti voi nella vita un po' di tifosità! Perché la vita è più bella quando ci si sente vive dentro.

\section*{Nostalgia - ricerca}
La nostalgia è un’emozione che tutti i tifosi della Ferrari conoscono bene. Sentiamo la nostalgia della vittoria. L’ultimo mondiale vinto dalla Ferrari è stato nel 2008. Da allora, ogni stagione ci porta speranza, sogni e qualche delusione. Ci sono stati momenti in cui sembrava possibile vincere, ma gli altri team erano sempre più forti. E così la nostalgia diventa una parte di noi, un sentimento che ci accompagna sempre.

La parola “nostalgia” ha un’origine interessante. Viene dal greco: \textit{“nóstos”} significa ritorno e \textit{“álgos”} significa dolore. Letteralmente vuol dire “il dolore del ritorno”. In passato, i medici usavano questa parola per parlare dei soldati svizzeri lontani da casa, che soffrivano molto. Oggi la nostalgia non è più una malattia, ma un’emozione normale. È il desiderio di ritrovare qualcosa che abbiamo amato e che non possiamo più avere.

La nostalgia non è tristezza. È anche speranza. Ci fa credere che un giorno torneremo a vincere. Ci ricorda le emozioni belle del passato e ci spinge a continuare a credere. La nostalgia ci fa apprezzare il presente e ci insegna a vivere con più attenzione i momenti importanti. Senza nostalgia, forse non sapremmo quanto sono preziosi i ricordi e i sogni.

Per esempio, penso alle gare della Ferrari che ho visto in televisione da piccolo. Ricordo il rumore dei motori, il cuore che batteva forte, l’emozione di un sorpasso. Ora, quando vedo le stesse gare, sento nostalgia: è un desiderio di tornare a quei momenti e sentire di nuovo quella felicità. Ma allo stesso tempo, questa nostalgia mi dà speranza: voglio che la Ferrari vinca ancora, voglio sentire di nuovo quelle emozioni con nuovi trionfi.

La nostalgia può anche unire le persone. Quando parlo con altri tifosi, condividiamo ricordi e sogni. Raccontiamo le gare, i piloti e le vittorie passate. Questo ci fa sentire parte di qualcosa di grande e ci aiuta a non perdere mai la passione. Anche se non vinciamo ogni anno, la nostalgia ci ricorda perché amiamo la Ferrari e perché continuiamo a seguire ogni gara.

\section*{Adagio - ricerca}
Ma la corsa non è la unica passione della mia vita. Mi piace anche la musica. E qui entra un’altra parola speciale: \textit{adagio}.

In musica, \textit{adagio} significa un tempo lento, calmo e rilassato. Viene dall’italiano “ad agio”, che vuol dire “con calma”. Non è solo una velocità: è un modo di sentire e di vivere. L’adagio ci invita a fermarci, respirare, osservare e ascoltare. Ci ricorda che non tutto deve essere veloce, e che a volte la calma porta più emozioni della fretta.

Ad esempio, quando ascolto un brano adagio, chiudo gli occhi e sento le note che scorrono lentamente. Ogni nota ha tempo di entrare dentro di me, di farmi pensare e di emozionarmi. Non voglio correre: voglio vivere ogni momento della musica.

In realtà, adagio e nostalgia sono collegati. La nostalgia ci insegna a ricordare, aspettare e sperare, con pazienza. Anche quando sembriamo fermi, dentro di noi ci sono emozioni e ricordi che continuano a muoversi, proprio come le note di un brano adagio. Anche nell’attesa e nella lentezza, c’è bellezza.

Sono tifoso della Ferrari, ma sono anche un amante dell’adagio. Sono due passioni diverse, ma hanno qualcosa in comune: il tempo. Il tempo dell’attesa, del ricordo, della speranza. Il tempo ci insegna a vivere e a sentire la vita con più profondità, e a trovare gioia anche nei momenti più lenti o difficili.

Ogni tanto penso che la vita sia un po’ come un gran premio e un brano musicale insieme. Ci sono momenti veloci, emozionanti e pieni di adrenalina, ma ci sono anche momenti lenti, in cui dobbiamo aspettare e riflettere. La nostalgia ci ricorda il passato, l’adagio ci insegna a vivere il presente. Insieme ci aiutano a guardare al futuro con pazienza e speranza.
\end{document}